\documentclass{beamer}

\pdfmapfile{+sansmathaccent.map}


\mode<presentation>
{
  \usetheme{Warsaw} % or try Darmstadt, Madrid, Warsaw, Rochester, CambridgeUS, ...
  \usecolortheme{crane} % or try seahorse, beaver, crane, wolverine, ...
  \usefonttheme{serif}  % or try serif, structurebold, ...
  \setbeamertemplate{navigation symbols}{}
  \setbeamertemplate{caption}[numbered]
} 


%%%%%%%%%%%%%%%%%%%%%%%%%%%%
% itemize settings

\definecolor{mypink}{RGB}{255, 30, 80}
\definecolor{myhotpink}{RGB}{255, 80, 200}
\definecolor{mywarmpink}{RGB}{255, 60, 160}
\definecolor{mylightpink}{RGB}{255, 80, 200}
\definecolor{mydarkpink}{RGB}{155, 25, 60}

\definecolor{myblue}{RGB}{240, 240, 255}
\definecolor{mydarkblue}{RGB}{60, 160, 255}

\definecolor{mygreen}{RGB}{0, 200, 0}
\definecolor{mygreen2}{RGB}{245, 255, 230}

\definecolor{mygray}{gray}{0.8}
\definecolor{mydarkgray}{gray}{0.4}

\setbeamertemplate{itemize items}[default]

\setbeamertemplate{itemize item}{\color{mywarmpink}$\blacksquare$}
\setbeamertemplate{itemize subitem}{\color{mydarkblue}$\blacktriangleright$}
\setbeamertemplate{itemize subsubitem}{\color{mygray}$\blacksquare$}



\setbeamercolor{palette quaternary}{fg=white,bg=mydarkpink}
\setbeamercolor{titlelike}{parent=palette quaternary}

\setbeamercolor{palette quaternary2}{fg=black,bg=mylightpink}
\setbeamercolor{frametitle}{parent=palette quaternary2}



\setbeamerfont{frametitle}{size=\Large,series=\scshape}
\setbeamerfont{framesubtitle}{size=\normalsize,series=\upshape}


%%%%%%%%%%%%%%%%%%%%%%%%%%%%
% block settings

\setbeamercolor{block title}{bg=red!50,fg=black}

\setbeamercolor*{block title example}{bg=mygreen!40!white,fg=black}

\setbeamercolor*{block body example}{fg= black,
bg= mygreen2}


%%%%%%%%%%%%%%%%%%%%%%%%%%%%
% URL settings
\hypersetup{
    colorlinks=false,
    linkcolor=blue,
    filecolor=blue,      
    urlcolor=blue,
}

%%%%%%%%%%%%%%%%%%%%%%%%%%

\renewcommand{\familydefault}{\rmdefault}

\usepackage{amsmath}
\usepackage{mathtools}


\usepackage{subcaption}

\newcommand{\bo}[1] {\mathbf{#1}}
\newcommand{\R} {\mathbb{R}}
\DeclareMathOperator*{\argmin}{arg\,min}


%%%%%%%%%%%%%%%%%%%%%%%%%%%%
% code settings

\usepackage{listings}
\usepackage{color}
% \definecolor{mygreen}{rgb}{0,0.6,0}
% \definecolor{mygray}{rgb}{0.5,0.5,0.5}
\definecolor{mymauve}{rgb}{0.58,0,0.82}
\lstset{ 
  backgroundcolor=\color{white},   % choose the background color; you must add \usepackage{color} or \usepackage{xcolor}; should come as last argument
  basicstyle=\footnotesize,        % the size of the fonts that are used for the code
  breakatwhitespace=false,         % sets if automatic breaks should only happen at whitespace
  breaklines=true,                 % sets automatic line breaking
  captionpos=b,                    % sets the caption-position to bottom
  commentstyle=\color{mygreen},    % comment style
  deletekeywords={...},            % if you want to delete keywords from the given language
  escapeinside={\%*}{*)},          % if you want to add LaTeX within your code
  extendedchars=true,              % lets you use non-ASCII characters; for 8-bits encodings only, does not work with UTF-8
  firstnumber=0000,                % start line enumeration with line 0000
  frame=single,	                   % adds a frame around the code
  keepspaces=true,                 % keeps spaces in text, useful for keeping indentation of code (possibly needs columns=flexible)
  keywordstyle=\color{blue},       % keyword style
  language=Octave,                 % the language of the code
  morekeywords={*,...},            % if you want to add more keywords to the set
  numbers=left,                    % where to put the line-numbers; possible values are (none, left, right)
  numbersep=5pt,                   % how far the line-numbers are from the code
  numberstyle=\tiny\color{mygray}, % the style that is used for the line-numbers
  rulecolor=\color{black},         % if not set, the frame-color may be changed on line-breaks within not-black text (e.g. comments (green here))
  showspaces=false,                % show spaces everywhere adding particular underscores; it overrides 'showstringspaces'
  showstringspaces=false,          % underline spaces within strings only
  showtabs=false,                  % show tabs within strings adding particular underscores
  stepnumber=2,                    % the step between two line-numbers. If it's 1, each line will be numbered
  stringstyle=\color{mymauve},     % string literal style
  tabsize=2,	                   % sets default tabsize to 2 spaces
  title=\lstname                   % show the filename of files included with \lstinputlisting; also try caption instead of title
}

%%%%%%%%%%%%%%%%%%%%%%%%%%%%
% tikz settings

\usepackage{tikz}
\tikzset{every picture/.style={line width=0.75pt}}

%%%%%%%%%%%%%%%%%%%%%%%%%%%%

\usepackage{qrcode}


\title{ Minimax }
\subtitle{Computational Intelligence, Lecture 13}
\author{by Sergei Savin}
\centering
\date{Spring 2021}



\begin{document}
\maketitle



\begin{frame}{Content}

\begin{itemize}
% \item Mixed Integer Linear Programming (MILP)
% \item Mixed Integer Quadratic Programming (MIQP)
% \item Remarks
% \item Example: Footstep planning
% \item Big-M method relaxation
% \begin{itemize}
%     \item Basic idea
%     \item Systems of inequalities
%     \item Illustration, part 1
%     \item Illustration, part 2
%     \item Multiple variables
% \end{itemize}
% \item  Example: Footstep planning
% \begin{itemize}
%     \item Formulation as MIQP
%     \item Evenly spaced steps
%     \item Code, part 1
%     \item Code, part 2
%     \item Code, part 3
% \end{itemize}
\item Homework
\end{itemize}

\end{frame}



\begin{frame}{MiniMax problems}
\framesubtitle{Example}
\begin{flushleft}

Consider the following problem: 

\begin{example}
Find smallest $x \in \R$, such that $x + y \geq 1$, where $|y| \leq 2$.
\end{example}

\bigskip

In that example we need to find optimal value of $x$ subject to a constraint where another unknown variable is present; the constraint has to be satisfied for the \emph{worst-case scenario}, in this case it is $y = -2$. Solution is $x = 3$

\bigskip

This is closely related to \emph{minimax optimization}
 
\end{flushleft}
\end{frame}





\begin{frame}{MiniMax: Linear constraint}
\framesubtitle{Part 1}
\begin{flushleft}

Consider the following problem:
%
\begin{equation}
\begin{aligned}
& \underset{\bo{x}}{\text{min}} \ \underset{\bo{y}}{\text{max}}
& & ||\bo{x}||, \\
& \text{subject to}
& & \bo{c}^\top \bo{x} + \bo{d}^\top \bo{y} \leq h, \\
& & & ||\bo{y}|| \leq p
\end{aligned}
\end{equation}

It is clear that worst-case scenario corresponds to the largest value of $\bo{d}^\top \bo{y}$, meaning that $\bo{y}$ should align with $\bo{d}$ and have its maximum possible length $p$. From that we conclude that $\bo{y} =  p \frac{\bo{d}}{|| \bo{d} ||} $.
 
\end{flushleft}
\end{frame}



\begin{frame}{MiniMax: Linear constraint}
\framesubtitle{Part 2}
\begin{flushleft}

Therefor $\bo{c}^\top \bo{x} + \bo{d}^\top \bo{y} \leq h$ becomes:

\begin{align}
    \bo{c}^\top \bo{x} + p  \frac{\bo{d}^\top\bo{d}}{|| \bo{d} ||} \leq h \\
    \bo{c}^\top \bo{x} + p || \bo{d} || \leq h 
\end{align}

\bigskip

Thus our problem becomes:
%
\begin{equation}
\begin{aligned}
& \underset{\bo{x}}{\text{min}}
& & ||\bo{x}||, \\
& \text{subject to}
& & \bo{c}^\top \bo{x} \leq h - p || \bo{d} ||
\end{aligned}
\end{equation}
 
\end{flushleft}
\end{frame}



\begin{frame}{MiniMax: Quadratic constraint, type 1}
\framesubtitle{Part 1}
\begin{flushleft}

Consider the following problem, where $\bo{x}^*$ is the desired value of $\bo{x}$:
%
\begin{equation}
\begin{aligned}
& \underset{\bo{x}}{\text{min}} \ \underset{\bo{y}}{\text{max}}
& & ||\bo{x} - \bo{x}^*||, \\
& \text{subject to}
& & \bo{y}^\top \bo{D} \bo{x} \leq h, \\
& & & ||\bo{y}|| \leq p
\end{aligned}
\end{equation}

This time worst-case scenario corresponds to $\bo{y}$ aligned with $\bo{D} \bo{x}$ and having its maximum possible length $p$. From that we conclude that $\bo{y} = p \frac{\bo{D} \bo{x}}{|| \bo{D} \bo{x} ||} $. Let us substitute it to $\bo{y}^\top \bo{D} \bo{x}$:

\begin{equation}
    p \left( \frac{\bo{D} \bo{x}}{|| \bo{D} \bo{x} ||} \right)^\top \bo{D} \bo{x} =
    p \frac{\bo{x}^\top \bo{D}^\top \bo{D} \bo{x}}{|| \bo{D} \bo{x} ||} = 
    p \frac{|| \bo{D} \bo{x} ||^2}{|| \bo{D} \bo{x} ||} = 
    p || \bo{D} \bo{x} ||
\end{equation}
 
\end{flushleft}
\end{frame}



\begin{frame}{MiniMax: Quadratic constraint, type 1}
\framesubtitle{Part 2}
\begin{flushleft}

Thus our problem becomes:
%
\begin{equation}
\begin{aligned}
& \underset{\bo{x}}{\text{min}}
& & ||\bo{x} - \bo{x}^*||, \\
& \text{subject to}
& & || \bo{D} \bo{x} || \leq \frac{h}{p}
\end{aligned}
\end{equation}

which is an SOCP.

\end{flushleft}
\end{frame}



\begin{frame}{MiniMax: Quadratic constraint, type 2}
\framesubtitle{Part 1}
\begin{flushleft}

A more general case of the previous problem is:
%
\begin{equation}
\begin{aligned}
& \underset{\bo{x}}{\text{min}} \ \underset{\bo{y}}{\text{max}}
& & ||\bo{x} - \bo{x}^*||, \\
& \text{subject to}
& & (\bo{y} - \bo{a})^\top \bo{D} (\bo{x} - \bo{b}) \leq h, \\
& & & ||\bo{y}|| \leq p
\end{aligned}
\end{equation}
%

We can rewrite $(\bo{y} - \bo{a})^\top \bo{D} (\bo{x} - \bo{b}) \leq h$ as:

\begin{equation}
    \bo{y}^\top \bo{D} (\bo{x} - \bo{b}) - \bo{a}^\top \bo{D} (\bo{x} - \bo{b}) \leq h
\end{equation}

With that we see that the worse case scenario is $\bo{y}$ is aligned with $\bo{D} (\bo{x} - \bo{b})$ and has length $p$:

\begin{equation}
    \bo{y} = p \frac{\bo{D} (\bo{x} - \bo{b})}{|| \bo{D} (\bo{x} - \bo{b}) ||}
\end{equation}

 
\end{flushleft}
\end{frame}




\begin{frame}{MiniMax: Quadratic constraint, type 2}
\framesubtitle{Part 2}
\begin{flushleft}

Then $\bo{y}^\top \bo{D} (\bo{x} - \bo{b}) - \bo{a}^\top \bo{D} (\bo{x} - \bo{b}) \leq h$ becomes:

\begin{equation}
    p \frac{ (\bo{x} - \bo{b})^\top \bo{D}^\top \bo{D} (\bo{x} - \bo{b}) }{|| \bo{D} (\bo{x} - \bo{b}) ||}  - \bo{a}^\top \bo{D} (\bo{x} - \bo{b}) \leq h
\end{equation}
%
which is the same as:

\begin{equation}
    p || \bo{D} (\bo{x} - \bo{b}) || - \bo{a}^\top \bo{D} (\bo{x} - \bo{b}) \leq h
\end{equation}
\begin{equation}
    || \bo{D} (\bo{x} - \bo{b}) ||  \leq \frac{1}{p}\bo{a}^\top \bo{D} (\bo{x} - \bo{b}) + \frac{h}{p}
\end{equation}

which is an SOCP constraint.
 
\end{flushleft}
\end{frame}



\begin{frame}{MiniMax: Quadratic constraint, type 2}
\framesubtitle{Part 2}
\begin{flushleft}

And thus we get:
%
\begin{equation}
\begin{aligned}
& \underset{\bo{x}}{\text{min}}
& & ||\bo{x} - \bo{x}^*||, \\
& \text{subject to}
& & || \bo{D} (\bo{x} - \bo{b}) ||  \leq \frac{1}{p}\bo{a}^\top \bo{D} (\bo{x} - \bo{b}) + \frac{h}{p}
\end{aligned}
\end{equation}
%
which is SOCP. 
 
\end{flushleft}
\end{frame}






\begin{frame}{MiniMax: Quadratic constraint, type 3}
\framesubtitle{Part 1}
\begin{flushleft}

A more general case of the previous problem is:
%
\begin{equation}
\begin{aligned}
& \underset{\bo{x}}{\text{min}} \ \underset{\bo{y}}{\text{max}}
& & ||\bo{x} - \bo{x}^*||, \\
& \text{subject to}
& & (\bo{y} - \bo{a})^\top \bo{D} (\bo{x} - \bo{b}) \leq h, \\
& & & ||\bo{H}\bo{y} + \bo{f}|| \leq p
\end{aligned}
\end{equation}
%
where $\bo{H}$ is has an inverse. We start by making substitution:
%and allows decomposition $\bo{H} = \bo{W}^\top \bo{W}$

\begin{equation}
    \bo{v} = \bo{H}\bo{y} + \bo{f}
\end{equation}
%
meaning $\bo{y} = \bo{H}^{-1}(\bo{v} -\bo{f})$:

\begin{align}
    (\bo{H}^{-1}(\bo{v} -\bo{f}) - \bo{a})^\top \bo{D} (\bo{x} - \bo{b}) \leq h \\
    \bo{v}^\top \bo{H}^{-\top} \bo{D} (\bo{x} - \bo{b}) - (\bo{H}^{-1}\bo{f} + \bo{a})^\top \bo{D} (\bo{x} - \bo{b}) \leq h\\
    \bo{v}^\top \bo{H}^{-\top} \bo{D} (\bo{x} - \bo{b}) - (\bo{H}\bo{a} + \bo{f})^\top \bo{H}^{-\top}\bo{D} (\bo{x} - \bo{b}) \leq h
\end{align}

 
\end{flushleft}
\end{frame}



\begin{frame}{MiniMax: Quadratic constraint, type 3}
\framesubtitle{Part 2}
\begin{flushleft}

We can introduce notation:
%
\begin{align}
    & \bo{M} = \bo{H}^{-\top} \bo{D} \\
    & \bo{g} = \bo{H}\bo{a} + \bo{f}
\end{align}
 %
With that we can re-write our constraint:
%
\begin{align}
    \bo{v}^\top \bo{M} (\bo{x} - \bo{b}) - \bo{g}^\top \bo{M} (\bo{x} - \bo{b}) \leq h \\
    (\bo{v} - \bo{g})^\top \bo{M} (\bo{x} - \bo{b}) \leq h 
\end{align}
%
And now we formulated type 3 problem as type 2:
%
\begin{equation}
\begin{aligned}
& \underset{\bo{x}}{\text{min}} \ \underset{\bo{v}}{\text{max}}
& & ||\bo{x} - \bo{x}^*||, \\
& \text{subject to}
& & (\bo{v} - \bo{g})^\top \bo{M} (\bo{x} - \bo{b}) \leq h, \\
& & & ||\bo{v}|| \leq p
\end{aligned}
\end{equation}

 
\end{flushleft}
\end{frame}



\begin{frame}{MiniMax: Quadratic constraint, type 4}
% \framesubtitle{Part 2}
\begin{flushleft}

Try solving this problem on your own:
%
\begin{equation}
\begin{aligned}
& \underset{\bo{x}}{\text{min}} \ \underset{\bo{y}}{\text{max}}
& & ||\bo{x} - \bo{x}^*||, \\
& \text{subject to}
& & (\bo{y} - \bo{a})^\top \bo{D} (\bo{x} - \bo{b})  + 
\bo{s}^\top \bo{y} + \bo{q}^\top \bo{x} \leq h, \\
& & & ||\bo{H}\bo{y} + \bo{f}|| \leq p
\end{aligned}
\end{equation}
 
\end{flushleft}
\end{frame}




\begin{frame}{Control with parameter uncertainty}
\framesubtitle{Part 1}
\begin{flushleft}

Consider the system:
%
\begin{equation}
    \dot{\bo{x}} = \bo{A}_p\bo{x} + \bo{B}_p\bo{u}
\end{equation}
%
where $\bo{A}_p$ and $\bo{B}_p$ are linear functions of parameters $\bo{p}$. We want to stabilize the origin.

\bigskip

Assume we use control law:
%
\begin{equation}
    \bo{u} = -\bo{K}\bo{x} + \bo{u}^*
\end{equation}

With that we get:
%
\begin{equation}
   \dot{\bo{x}} = (\bo{A}_p - \bo{B}_p\bo{K})\bo{x} + \bo{B}_p\bo{u}^*
\end{equation}


% \begin{equation}
%     \bo{u}^* = -\bo{B}^+(\bo{A}\bo{x}^* + \bo{c})
% \end{equation}
 
\end{flushleft}
\end{frame}



\begin{frame}{Control with parameter uncertainty}
\framesubtitle{Part 2}
\begin{flushleft}

Let us write Lyapunov function for the system:
%
\begin{align}
    & V = \bo{x}^\top \bo{S} \bo{x} \\
    & \dot{V} = \dot{\bo{x}}^\top \bo{S} \bo{x} + \bo{x}^\top \bo{S} \dot{\bo{x}} 
    = \\
    & = \bo{x}^\top \bo{S}(\bo{A}_p - \bo{B}_p\bo{K})\bo{x}
    + \bo{x}^\top (\bo{A}_p - \bo{B}_p\bo{K})^\top
    \bo{S} \bo{x}
    + \\
    & + \bo{x}^\top \bo{S} \bo{B}_p \bo{u}^*
    +
    \bo{u}^{*\top} \bo{B}_p^\top \bo{S} \bo{x}
\end{align}
%
Let us define:
%
\begin{align}
    & a = 2\bo{x}^\top \bo{S}(\bo{A}_p - \bo{B}_p\bo{K})\bo{x} \\
    & \bo{b} = 2\bo{x}^\top \bo{S} \bo{B}_p
\end{align}
%
With that we can find Jacobians:
%
\begin{align}
    & \bo{a}_x = \frac{\partial a}{\partial \bo{p}} &
    & \bo{B}_x = \frac{\partial \bo{b}}{\partial \bo{p}}
\end{align}

\end{flushleft}
\end{frame}




\begin{frame}{Control with parameter uncertainty}
\framesubtitle{Part 3}
\begin{flushleft}

Thus we get minimax constraint on the Lyapunov function
%
\begin{align}
\dot{V} = \bo{a}_x^\top \bo{p}_t + 
\bo{u}^{*\top} \bo{B}_x \bo{p}_t
\end{align}
%
where $\bo{p}_t$ are true values of parameters $\bo{p}$. Assuming:
%
\begin{align}
& \bo{p}_t = \bo{p} + \bo{p}_0
\end{align}
%
we get:

\begin{align}
& \dot{V} = \bo{a}_x^\top (\bo{p} + \bo{p}_0) + 
\bo{u}^{*\top} \bo{B}_x (\bo{p} + \bo{p}_0)
\end{align}
%
which is a minimax constraint. Let's solve it for the case $||\bo{p}|| \leq 1$.
%and $||\bo{G}\bo{p}|| \leq 1$

\end{flushleft}
\end{frame}



\begin{frame}{Control with parameter uncertainty}
\framesubtitle{Part 4}
\begin{flushleft}
Taking derivative of $\dot{V}$ with respect to $\bo{p}$ we get 
%
\begin{align}
& \frac{\partial \dot{V}}{\partial  \bo{p}}  = \bo{a}_x^\top  + 
\bo{u}^{*\top} \bo{B}_x
\end{align}
%
this is the direction where the function grow the most. But we know its length is 1, so we conclude that:

\begin{align}
& \bo{p}  = \frac{\bo{a}_x^\top  + 
\bo{u}^{*\top} \bo{B}_x}{|| \bo{a}_x^\top  + 
\bo{u}^{*\top} \bo{B}_x ||}
\end{align}

So:
%
\begin{align}
& \dot{V} = 
|| \bo{a}_x^\top  + 
\bo{u}^{*\top} \bo{B}_x ||
+
(\bo{a}_x^\top + 
\bo{u}^{*\top} \bo{B}_x) \bo{p}_0
\end{align}

\end{flushleft}
\end{frame}




\begin{frame}{Elliptical parameter uncertainty}
\framesubtitle{Part 1}
\begin{flushleft}

Let's do the same, but for the case when $||\bo{G}\bo{p}|| \leq 1$:

\begin{equation}
    \begin{cases}
\dot{V} = \bo{a}_x^\top (\bo{p} + \bo{p}_0) + 
\bo{u}^{*\top} \bo{B}_x (\bo{p} + \bo{p}_0) \leq 0 \\
||\bo{G}\bo{p}|| \leq 1
    \end{cases}
\end{equation}

First step is to introduce new variable:

\begin{align}
& \rho = \bo{G}\bo{p}
\end{align}
%
from which it follows that $\bo{p} = \bo{G}^{-1} \rho$ ($\bo{G}$ should be invertible for the parameters to be bounded). Hence we get:

\begin{equation}
    \begin{cases}
\dot{V} = \bo{a}_x^\top (\bo{G}^{-1} \rho + \bo{p}_0) + 
\bo{u}^{*\top} \bo{B}_x (\bo{G}^{-1} \rho + \bo{p}_0) \leq 0 \\
||\rho|| \leq 1
    \end{cases}
\end{equation}

\end{flushleft}
\end{frame}



\begin{frame}{Elliptical parameter uncertainty}
\framesubtitle{Part 2}
\begin{flushleft}

We can find gradient:

\begin{align}
& \frac{\partial \dot{V}}{\partial  \bo{p}}  = 
\bo{a}_x^\top \bo{G}^{-1}  + 
\bo{u}^{*\top} \bo{B}_x \bo{G}^{-1}
\end{align}

We know that length of $\rho$ is bounded, so:

\begin{align}
& \rho  = 
\frac{\bo{a}_x^\top \bo{G}^{-1}  + 
\bo{u}^{*\top} \bo{B}_x \bo{G}^{-1}}{|| \bo{a}_x^\top \bo{G}^{-1}  + 
\bo{u}^{*\top} \bo{B}_x \bo{G}^{-1} ||}
\end{align}

And thus we get SOCP constraint:

\begin{equation}
\dot{V} = 
|| \bo{a}_x^\top \bo{G}^{-1}  + 
\bo{u}^{*\top} \bo{B}_x \bo{G}^{-1} ||
+ 
\bo{a}_x^\top \bo{p}_0 + 
\bo{u}^{*\top} \bo{B}_x \bo{p}_0 \leq 0 
\end{equation}


\end{flushleft}
\end{frame}


% \begin{frame}{Control with parameter uncertainty}
% \framesubtitle{Part 2}
% \begin{flushleft}

% Assume the model depends on parameters $\bo{p}$:
% %
% \begin{equation}
%     \bo{c} = \bo{C}\bo{p}
% \end{equation}
 
% Then feed-forward component also depends on $\bo{p}$:
 
% \begin{equation}
%     \bo{B}\bo{u}^* = -(\bo{A}\bo{x}^* + \bo{C}\bo{p})
% \end{equation}

% Now, assume we don't know $\bo{p}$, but only have an estimate $\Bar{\bo{p}} = \bo{p} + \Tilde{\bo{p}}$, where $\Tilde{\bo{p}}$ is parameter estimation error:

% \begin{align}
%     \bo{B}\Bar{\bo{u}}^* = -(\bo{A}\bo{x}^* + \bo{C}\bo{p} + \bo{C}\Tilde{\bo{p}}) \\
%     \bo{B}\Bar{\bo{u}}^* = \bo{B}\bo{u}^* - \bo{C}\Tilde{\bo{p}}
% \end{align}

% \end{flushleft}
% \end{frame}




% \begin{frame}{Control with parameter uncertainty}
% \framesubtitle{Part 3}
% \begin{flushleft}

% Let us use control law $\bo{u} = -\bo{K} (\bo{x} - \bo{x}^*) + \Bar{\bo{u}}^*$. We get:

% \begin{align}
%     & \dot{\bo{x}} = \bo{A}\bo{x} - \bo{B}(\bo{K} (\bo{x} - \bo{x}^*) + \Bar{\bo{u}}^*) + \bo{c} \\
%     & \dot{\bo{x}} = \bo{A}\bo{x} - \bo{B} \bo{K} (\bo{x} - \bo{x}^*) + \bo{B}\Bar{\bo{u}}^* + \bo{c}
% \end{align}
 
% Let us define control error: $\bo{e} = \bo{x} - \bo{x}^*$ and write error dynamics by subtracting $\bo{A}\bo{x}^* + \bo{B}\bo{u}^* + \bo{c}$ from the closed-loop equation:

% \begin{align}
%     & \dot{\bo{e}} = \bo{A}\bo{e} - \bo{B} \bo{K} (\bo{x} - \bo{x}^*) + \bo{B}(\Bar{\bo{u}}^* - \bo{u}^*) \\
%     & \dot{\bo{e}} = \bo{A}\bo{e} - \bo{B} \bo{K} \bo{e} + \bo{B}\bo{u}^* - \bo{C}\Tilde{\bo{p}} - \bo{B}\bo{u}^* \\
%     & \dot{\bo{e}} = (\bo{A} - \bo{B}\bo{K}) \bo{e} - \bo{C}\Tilde{\bo{p}}
% \end{align}



% \end{flushleft}
% \end{frame}



% \begin{frame}{Control with parameter uncertainty}
% \framesubtitle{Part 4}
% \begin{flushleft}

% Next steps:

% \begin{enumerate}
%     \item Expand brackets.
%     \item Write Lyapunov
%     \item Write MiniMax
% \end{enumerate}



% \end{flushleft}
% \end{frame}












% \begin{frame}{Homework}
% % \framesubtitle{Parameter estimation}
% \begin{flushleft}


% \end{flushleft}
% \end{frame}





\begin{frame}
\centerline{Lecture slides are available via Moodle.}
\bigskip
\centerline{You can help improve these slides at:}
\centerline{
\textcolor{blue}{\href{https://github.com/SergeiSa/Computational-Intelligence-Slides-Spring-2021}{github.com/SergeiSa/Computational-Intelligence-Slides-Spring-2021}}
}
\bigskip

\textcolor{black}{\qrcode[height=1.5in]{https://git.io/JYRBT}}
\bigskip

\centerline{Check Moodle for additional links, videos, textbook suggestions.}
\end{frame}



\end{document}
